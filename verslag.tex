%Groeps verslag van TinLab AA
\documentclass{article}
\usepackage{graphicx} 
\usepackage[dutch]{babel}

\begin{document}
\sffamily
\begin{titlepage}
  \centering
    \vfill
    {\bfseries\Huge
      Verslag Tinlab Advanced Algorithms \\
        \vskip2cm
      }
      {\bfseries\Large
        A. J. Ruigrok \& J. I. Weverink\\
      }
      {
        \bfseries\normalsize
        176-671\\
        \vskip1cm
        \today\\
    }
    \vfill
    \includegraphics[width=4cm]{logohr.png} % also works with logo.pdf
    \vfill
    \vfill
\end{titlepage}
\newpage
\tableofcontents

\newpage
\section{Inleiding}
Voor het vak Advanced Algorithms wordt een sluis gemodelleerd. Wij hebben gekozen voor een shutsluis. Mede omdat we bij deze soort sluis bij schippers kunnen aankloppen voor vragen en informatie. Voor het modelleren van de sluis is onderzoek gedaan naar een aantal eigenschappen die de sluis heeft.
\\\\
Zie hier een referentie naar Royce~\cite{royce1987managing} en nog een naar Clarke~\cite{modelchecking}\ldots 


\section{Requirements}
De sluis moet aan de volgende requiriments voldoen.
De sluis moet 

\subsection{Veiligheid}
\section{Veiligheid}
Aan welke veiligheids eigenschappen moet een sluis zich voldoen
\subsection{subSection 2.1}

\section{Capaciteit}
Wat is de capaciteit van een sluis? Hoeveelheid water wat kan worden geaccepteerd en overpomp snelheid.
\subsection{subSection 3.1}

\section{Efficientie}
\subsection{subSection 4.1}

\section{Duurzaamheid}
\subsection{subSection 5.1}



\newpage

\newpage
\bibliography{references}
\bibliographystyle{plain}
\end{document}


