%Groeps verslag van TinLab AA
\documentclass{article}
\usepackage{graphicx} 
\usepackage[dutch]{babel}
\usepackage{verbatim} % for multi line comments

\begin{document}
\sffamily
\begin{titlepage}
  \centering
    \vfill
    {\bfseries\Huge
      Verslag Tinlab Advanced Algorithms \\
        \vskip2cm
      }
      {\bfseries\Large
        A. J. Ruigrok \& J. I. Weverink\\
      }
      {
        \bfseries\normalsize
        176-671\\
        \vskip1cm
        \today\\
    }
    \vfill
    \includegraphics[width=4cm]{logohr.png} % also works with logo.pdf
    \vfill
    \vfill
\end{titlepage}
\newpage
\tableofcontents

\newpage
\section{Inleiding}
Voor het vak Advanced Algorithms wordt een sluis gemodelleerd. Wij hebben gekozen voor een shutsluis. Mede omdat we bij deze soort sluis bij schippers kunnen aankloppen voor vragen en informatie. Voor het modelleren van de sluis is onderzoek gedaan naar een aantal eigenschappen die de sluis heeft.
\\\\
Zie hier een referentie naar latex Companion~\cite{latexcompanion} en nog een naar Einstein~\cite{einstein}\ldots 

\section{Literatuur onderzoek} %resultaten literatuur onderzoek
\textbf{Veiligheid}\\
Aan welke veiligheids eigenschappen moet een sluis zich voldoen
De kamer moet goed afgesloten zijn (waterdicht).
\begin{itemize}
\item Het water niveau moet worden gemonitoord.
\item Hoeveel water in \& hoeveel water uit gaat.
\end{itemize}

\textbf{Capaciteit}\\
Wat is de capaciteit van een sluis? Hoeveelheid water wat kan worden geaccepteerd en overpomp snelheid.\\
In de kamer moet minimaal ruimte zijn voor een plezier boot ca.10m, maar dit is niet relevant voor het modelleren.

\textbf{Efficientie}\\

\textbf{Duurzaamheid}\\
\section{Requirements} %Requirements van het te bouwen systeem
De sluis moet aan de volgende requirements voldoen.

\section{Specificaties} %Specificaties van het te bouwen systeem
Harde eisen die gemeten kunnen worden.

\section{Ontwerpen} % Het model gemaakt in UPPAAL

\section{Verificaties}
\subsection{Temporeel logische constructies}
\subsection{Resultaten}



\clearpage




\begin{comment}
A watertight chamber connecting the upper and lower canals, and large enough to enclose one or more boats. The position of the chamber is fixed, but its water level can vary.\\
A gate (often a pair of "pointing" half-gates) at each end of the chamber. A gate is opened to allow a boat to enter or leave the chamber; when closed, the gate is watertight.\\
A set of lock gear to empty or fill the chamber as required. This is usually a simple valve (traditionally, a flat panel (paddle) lifted by manually winding a rack and pinion mechanism) which allows water to drain into or out of the chamber; larger locks may use pumps.
\end{comment}




\newpage

\newpage
\bibliography{references} % DIT ZIJN VOORBEELDEN
\begin{thebibliography}{9}
\bibitem{latexcompanion} 
Michel Goossens, Frank Mittelbach, and Alexander Samarin. 
\textit{The \LaTeX\ Companion}. 
Addison-Wesley, Reading, Massachusetts, 1993.

\bibitem{einstein} 
Albert Einstein. 
\textit{Zur Elektrodynamik bewegter K{\"o}rper}. (German) 
[\textit{On the electrodynamics of moving bodies}]. 
Annalen der Physik, 322(10):891–921, 1905.

\bibitem{knuthwebsite} 
Knuth: Computers and Typesetting,
\\\texttt{http://www-cs-faculty.stanford.edu/\~{}uno/abcde.html}
\end{thebibliography}

\end{document}


