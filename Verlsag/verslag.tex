%Groeps verslag van TinLab AA
\documentclass{article}
\usepackage{graphicx} 
\usepackage[dutch]{babel}
\usepackage{verbatim} % for multi line comments

\begin{document}
\graphicspath{{../Images/}}
\sffamily
\begin{titlepage}
  \centering
    \vfill
    {\bfseries\Huge
      Verslag Tinlab Advanced Algorithms \\
        \vskip2cm
      }
      {\bfseries\Large
        A. J. Ruigrok \& J. I. Weverink\\
      }
      {
        \bfseries\normalsize
        176-671\\
        \vskip1cm
        \today\\
    }
    \vfill
    \includegraphics[width=4cm]{logohr.png} % also works with logo.pdf
    \vfill
    \vfill
\end{titlepage}
\newpage
\tableofcontents

\newpage
\section{Inleiding}
Voor het vak Advanced Algorithms wordt een sluis gemodelleerd. Wij hebben gekozen voor een shutsluis. Mede omdat we bij deze soort sluis bij schippers kunnen aankloppen voor vragen en informatie. Voor het modelleren van de sluis is onderzoek gedaan naar een aantal eigenschappen die de sluis heeft.
\\\\
Zie hier een referentie naar latex Companion~\cite{latexcompanion} en nog een naar Einstein~\cite{einstein}\ldots 

\clearpage

\section{Literatuur onderzoek} %resultaten literatuur onderzoek
\subsection{Veiligheid}
Aan welke veiligheids eigenschappen moet een sluis zich voldoen
De kamer moet goed afgesloten zijn (waterdicht).
\begin{itemize}
\item Het water niveau moet worden gemonitoord.
\item Hoeveel water in \& hoeveel water uit gaat.
\end{itemize}

% Richtlijnen Vaarwegen 2017
% https://waterrecreatienederland.nl/content/uploads/2018/04/richtlijnen-vaarwegen-2017.pdf

\subsubsection{Invloed van wind} % pagina 70
Wind kan ervoor zorgen dat boten onbedoeld in beweging worden gebracht. In de huidige tijd kunnen we ervan uitgaan dat de meeste beroepsschepen over een boegschroef  met voldoende vermogen beschikken. De schepen kunnen d.m.v de boegschroef de werking van de wind compenseren. Tevens is het de taak van de schipper om de werking van de wind te anticiperen. De schipper moet het schip op zijn plek houden zodra de sluis wordt benadert of gedurende het gebruik van de sluis.

Hieruit kan worden vernomen dat er geen rekening gehouden hoefd te worden met windkrachten in het model. De verantwoordelijkheid om hiermee juist te handelen wordt bij de schipper gelegd. Dit is dan ook een menselijke handeling en zal niet worden verwerkt in het model\cite{rijkswaterstaat2017}

\vskip0.5cm

\subsubsection{Vorming van ijs}
Ijsvorming kan een probleem vormen voor een sluis. Ijs kan ervoor zorgen dat de deuren niet meer goed functioneren. Elk type deur heeft zijn eigen gevoeligheden:
\begin{itemize}
\item Puntdeuren zijn gevoelig voor ijsvorming en ophoping in en direct voor de deurkassen. Dit zorgt ervoor dat de deuren niet open kunnen gaan.
\item Enkelvoudige draaideuren zijn ook gevoelig voor ijsvorming voor en ophoping in en direct voor de deurkassen.
\item Roldeuren kunnen vastlopen in drijvend ijs, dat zich ophoopt. Dit kan leiden tot vastvriezen.
\item Hefdeuren zijn gevoelig voor aanhechting van ijs. Aangehecht ijs kan op onderdoorvarende schepen kan vallen. Ook kan de deur klemlopen of dusdanig zwaar maken dat de deur te zwaar wordt om te heffen.
\end{itemize}

Er zijn natuurlijk ook middelen om de sluis ijsvrij te maken. Deze middelen zijn: luchtbellenscherm, kasblaasinstallatie, verwaringselementen of een gesloten constructie.

\vskip0.5cm

{\large \textbf{KEUZE MAKEN WEL OF NIET MODELLEREN}}

\subsection{Capaciteit}
Wat is de capaciteit van een sluis? Hoeveelheid water wat kan worden geaccepteerd en overpomp snelheid.\\
In de kamer moet minimaal ruimte zijn voor een plezier boot ca.10m, maar dit is niet relevant voor het modelleren.

\vskip0.5cm

\subsection{Efficientie}

\subsection{Duurzaamheid}

\section{Requirements} %Requirements van het te bouwen systeem
De sluis moet aan de volgende requirements voldoen.
\vskip0.5cm
De volgende onderwerpen vallen buiten de scope van het modelleren. Hier hoeft dus geen rekening meegehouden te worden.
\begin{itemize}
\item Windkrachten
\end{itemize}

\section{Specificaties} %Specificaties van het te bouwen systeem
Harde eisen die gemeten kunnen worden.

\section{Ontwerpen} % Het model gemaakt in UPPAAL

\section{Verificaties}
\subsection{Temporeel logische constructies}
\subsection{Resultaten}



\clearpage




\begin{comment}
A watertight chamber connecting the upper and lower canals, and large enough to enclose one or more boats. The position of the chamber is fixed, but its water level can vary.\\
A gate (often a pair of "pointing" half-gates) at each end of the chamber. A gate is opened to allow a boat to enter or leave the chamber; when closed, the gate is watertight.\\
A set of lock gear to empty or fill the chamber as required. This is usually a simple valve (traditionally, a flat panel (paddle) lifted by manually winding a rack and pinion mechanism) which allows water to drain into or out of the chamber; larger locks may use pumps.
\end{comment}




\newpage

\newpage
\bibliography{references} % DIT ZIJN VOORBEELDEN
\begin{thebibliography}{9}
\bibitem{rijkswaterstaat2017}
Rijkswaterstaat (2017).
\textit{Richtlijnen Vaarwegen 2017}\\
% Editors. Publisher.

\bibitem{latexcompanion} 
Michel Goossens, Frank Mittelbach, and Alexander Samarin. 
\textit{The \LaTeX\ Companion}. 
Addison-Wesley, Reading, Massachusetts, 1993.

\bibitem{einstein} 
Albert Einstein. 
\textit{Zur Elektrodynamik bewegter K{\"o}rper}. (German) 
[\textit{On the electrodynamics of moving bodies}]. 
Annalen der Physik, 322(10):891–921, 1905.

\bibitem{knuthwebsite} 
Knuth: Computers and Typesetting,
\\\texttt{http://www-cs-faculty.stanford.edu/\~{}uno/abcde.html}
\end{thebibliography}

\end{document}


% book:
% author (release year). \textit{title}\\
% Editors. Publisher. 
